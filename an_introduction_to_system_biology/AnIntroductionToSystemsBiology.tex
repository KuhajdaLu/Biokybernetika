\documentclass[11pt,a4paper]{report}
\usepackage{graphicx}
\usepackage[czech]{babel}
\usepackage[utf8]{inputenc}
\usepackage{titling}
\usepackage{gensymb}
\usepackage{pdfpages}
\usepackage{mathtools}
\usepackage{xcolor}
\usepackage{multirow}
\usepackage{caption}
\usepackage{float}
\usepackage{pdfpages}
\usepackage{subcaption}
\usepackage{enumitem}
\usepackage{listings}
\usepackage{amsmath}
\usepackage{amssymb}
\usepackage[nopar]{lipsum}
\setlength\parindent{1cm}
\usepackage{hyperref}
\setlength{\hoffset}{-1.8cm} 
\setlength{\voffset}{-2cm}
\setlength{\textheight}{24.0cm} 
\setlength{\textwidth}{17cm}
\usepackage{titlesec}

\setcounter{secnumdepth}{4}




\begin{document}
\begin{huge}
\title{\Huge{An Introduction To Systems Biology\\Design Principles of Biological Circuits\\Uri Alon}}
\author{Lukáš Kuhajda}
\date{2019/2020}
\maketitle
\end{huge}

\newpage

\chapter{Transkripční síť}
- tranksripce (přepis) - výroba RNA dle informace z DNA\\
- transkripční síť - interakce mezi transkripčními faktory a geny\\
- buňka - tisíce proteinů - každý svůj úkol - vysoká přesnost\\
\indent - sledování prostředí, produkce co je třeba v potřebném množství - prováděno \textbf{transkripční sítí}\\

\section{Kognitivní problémy buňky}
- kognitivní - poznávací, vjemový\\
- rozeznávání tepla, tlaku, signály z jiných buněk, prospěšné živiny, škodlivé látky..., vnitřního stavu (poškození DNA, proteinů, membrány)\\
- \textbf{transkripční faktory} - reprezentace stavu prostředí\\
\indent - pro rychlý přechod mezi aktivními a psivními molekulovými stavy\\
\indent - aktivní trans.fak. se může vázat na DNA a regulovat rychlost čtení cílových genů\\
\indent - čtení genů do mRNA $\rightarrow$ přeložení do proteinu (\textbf{produkt genu}) $\rightarrow$ působení na prostředí\\

\section{Prvky transkripční sítě}
- gen - úsek DNA - sekvence kóduje informaci pro produkci proteinu\\
\indent - transkripce genu - \textbf{RNA polymeráza} tvoří mRNA odpovídající kódovací sekvenci genu\\  
\indent - \textbf{promotor} - kontrola rychlosti přepsání genu, množství mRNA za jednotku času\\
\indent \indent - regulační oblast DNA předcházející genům\\
\indent \indent - vazba RNAp s daným místem v promotoru - kvalita místa úměrná s transk. mírou\\
\indent - RNAp - jednání se všemi geny, změny ale prováděny z jednotlivých genech transkr. faktory\\
\indent \indent - vazbou transkr.fak. s místem v promotoru cílového genu dochází k ovlivnění trans.míry \\
\indent \indent \indent - ovlivnění kdy RNAp zahájí transkripci genu\\
\indent - transkripční faktor \\
\indent \indent - \textbf{aktivátor} - zvyšuje míru transkripce genu\\
\indent \indent - \textbf{represor} - nepřepisování genu v RNA, redukce míry transkripce\\
\indent - \textbf{transkripční síť} - proteiny transk.fak. kódovány geny, které jsou regulovány jinými trans.fak.\\
\indent \indent - popisuje všechny regulační transkripční interakce v buňce\\
\indent \indent - \textbf{nody} - uzly v grafu, geny\\
\indent \indent - \textbf{hrany} - transkripční regulace genu produktem proteinu jiného genu\\
\indent \indent - \textbf{signály} - vstup do sítě\\
\indent \indent \indent - molekuly, modifikace proteinu - přímé ovlivnění aktivitu transkr.fak.\\
\indent \indent \indent - často způsobení fyzické změny tvaru proteinu transkr.fak.\\
\indent \indent \indent \indent - převzetí aktivního molekulového stavu\\
\indent - signál $\rightarrow$ změna aktivity transk.fak. $\rightarrow$ změna produkce proteinů\\
\indent - některé proteiny jsou transk.fak., které aktivují další geny\\
\indent - zbytek vykonává různé funkce v buňce - budování struktury, katalyzace \\

\subsection{Oddělění časových intervalů}
- vázání transkr.fak. k dané části DNA v rámci sekund\\
- transkripce a translace cílového genu v rámci minut\\
- hromadění proteinového produktu v rámci minut až hodin\\
- úroveň aktivity trans.fak. považována za stabilní v rámci rovnic popisujících dynamiku pro pomalý časový rozvrh změn úrovně proteinů\\
- velká škála mechanismů, jak transkr.fak. regulují geny\\
- při spojení transkr.fak. s DNA se doplní o RNAp pro kontrolu produkce mRNA\\
- transkripční síť - modularita komponentů\\
\indent - možnost vzít DNA z genu z jednoho do druhého organismu\\
\indent \indent - síť tvárná - evoluce, snadné začlenění nových genů a regulací\\
- GFP - green fluorescent protein - světélkování u medúz\\
- hrany v síti se jeví, že se vyvíjí v rychlejších intervalech než kódování částí genů\\
\indent - myši, lidi - podobné geny, rozdíl transkripční regulaci genů\\
\indent \indent - rozdíl v druzích ne tolik v genech, ale v hranách transrk.sítě\\

\subsection{Označení hran: aktivátory a represory}
- aktivátor - pozitivní kontrola, vzrůst míry transkripce, node obsahuje více + hran\\
- represor - negativní kontrola, redukce míry transkripce, node obsahuje více - hran\\
- síť většinou obsahuje více pozitivních hran - 60-80$\%$\\
- většina aktivátorů funguje za určitých podmínek jako represory pro dané cílové geny\\
\indent - platí i naopak\\
- duální transkr.fak - při splnění podmínky působí na gen jako aktivátor, při jiné jako pasivátor\\
- transkr.fak. mají tendenci použít stejný mód regulace pro většinu jejich cílových genů\\
\indent - označení hran jdoucích do nodu (transkrypční interakce regulující gen) nejsou tak korelované\\
- označení hran odcházejících bývaá spíše korelované, než označení hran příchozích\\

\subsection{Čísla na hranách: vstupní funkce}
- označení síly interakce\\
- \textbf{vstupní funkce} - síla účinku transk.fak. na míru transkripce cílového genu\\
- $X\rightarrow Y$ ... $rate$ $of$ $production$ $Y=f(X^*)$ - míra produkce proteinu $Y$ za jednotku času\\
\indent - $X^*$ - aktivní forma koncetrace $X$
\indent - $f(X^*)$ obvykle monotóní, tvar $S$ - rostoucí když $X$ aktivátor a naopak\\
- Hillova funkce - popisuje mnoho vsupních funkcí genů\\
\indent - pro aktivátor\\
\begin{equation}
	f(X^*)=\frac{\beta X^{*n}}{K^n+N^{*n}}
\end{equation}
\indent \indent K ... aktivační koeficient - udává se v koncentraci $X$ nutnou pro aktivaci výrazu\\
\indent \indent \indent - souvisí se slučivostí mezi $X$ a jeho umístěním na promotoru, + další faktory\\
\indent \indent $\beta$ - maximální hodnota výrazu, $X^*>>K$\\
\indent \indent \indent - při vysoké koncentraci se $X^*$ váže na promotors vysokou ppstí a stimuluje RNAp\\
\indent \indent \indent k produkci velkého množství mRNA za jednotku času\\
\indent \indent n - \textbf{Hillův koeficient} - řídí šikmost vstupní funkce\\
\indent \indent \indent - obvykle 1-4, čím vyšší tím šikmější\\
\indent - pro represor
\begin{equation}
	f(X^*)=\frac{\beta}{1+(\frac{X^*}{K})^n}
\end{equation}
- každá hrana tedy může nést 3 čísla, změny při evoluci\\
- spousta genů má nenulový počáteční stav hodnoty výrazu - \textbf{základní hodnota výrazu} - $\beta_0$\\

\subsection{Logické vstupní funkce: jednoduchý framwork k pochopení dynamiky sítě}
- Hill dobrý pro detailní modely, z matematického hlediska lepší jednodušší funkce pro zachycení základního chování\\
- \textbf{logická aproximace} - častá aproximace vstupní funkce v transkripční síti, skoková\\
\indent - pro jednoduché grafické řešení dynamických rovnic \\
\indent - gen buď zapnutý (ON - $f(X^*)=\beta$)/ vypnutý (OFF - $f(X^*)=0$), aktivační práh = $K$\\
\indent - pro aktivátor\\
\begin{equation}
	f(X^*)=\beta\theta(X^*>K)
\end{equation}
\indent \indent $\theta$ ... 0/1
\indent - pro represor obrácené logické znaménko\\

\subsection{Vícerozměrové vstupní funkce řídící geny s více vstupy}
- sousta nodů má více než 1 vstupní hranu\\
- aktivita promotoru je vícerozměrné vstupní funkce různých transk.fak.
- můžou být aproximovány logickými funkcemi\\
- geny co pro aktivaci potřebují oba aktivátory proteinu vázané na promotor\\
\begin{equation}
	f(X^*,Y^*)=\beta\theta(X^*>K_x)\theta(Y^*>K_y)\sim X^*AND^{.}Y^*
\end{equation}
- alespoň jeden\\
\begin{equation}
f(X^*,Y^*)=\beta\theta(X^*>K_xOR^.Y^*>K_y)\sim X^*OR^{.}Y^*
\end{equation}
- některé jako suma vstupů\\
\begin{equation}
f(X^*,Y^*)=\beta_x X^*+\beta_y Y^*
\end{equation}
- jednoduchá změna funkční formy vstupní funkce mutací v promotoru regulovaného genu\\
\indent - změna z AND na OR během pár mutací promotoru v $lac$ (E.coli)\\

\subsection{Prozatimní shrnutí}
- transkripční síť popisuje transkripční regulaci genů\\
- node reprezentuje gen, hrana z X $\rightarrow$ Y - gen X se kóduje pro trans.fak proteinu, který se váže na promotor genu Y, upravuje míru transkripce\\
\indent - protein kódovaný genem X mění míru produkce proteinu kódovaného genem Y, Z pak zas může být transk.fak. genu Z - vytváření sítě\\
- většina nodů značí geny, které nejsou transk.fak, ale nesou různé funkce buňky\\
- vstupy do sítě jsou informace o prostředí, mění příslušné transkr.fak.\\
- aktivní transkr.fak. se vážou na daná místa DNA promotoru cílového genu - regulace míry transkripce\\
\indent - míra produkce produktu genu Y je funkce koncentrace aktvního transkr.fak. $X^*$\\
\indent - geny regulované více transkr.fak. ají vícerozměrovou vstupní funkci - Hill, logické\\
- hrany a vstupní funkce jsou pod tlakem výběru, nevyužívaná hrana zanikne mutacemi\\
\indent - změna jednoho nebo pár písmen v oblasti DNA, kde se váže X v promotoru Y k zrušení hrany X $\rightarrow$ Y

\section{Dynamika a časová odezva jednoduché regulace genu}
- 1 hrana sítě, gen regulovaný 1 regulátorem bez dalších vstupů (nebo jen konstanty)\\
- $X\rightarrow Y$ - transk.fak. X reguluje gen Y\\
- bez vstupu X neaktivní a Y není produkováno\\
- objevení signálu $S_x$, okamžitá přeměna $X$ do aktivní formy $X^*$ a navázání na promotor genu $Y$\\
\indent $\rightarrow$ přepis genu $Y$, překlad mRNA $\rightarrow$ hromadění proteinu $Y$\\
\indent - buňka produkuje protein Y v konstatní míře - $\beta$ - jednotky koncentrace za jednotku času\\
- vyvážení produkce proteinu \\
\indent \textbf{degradace} - specifická destrukce speciálními proteiny v buňce\\
\indent \indent - míra degradace - $\alpha_{deg}$\\
\indent \textbf{ředění} - redukce koncentrace v důsledku zvýšení objemu buněk během růstu\\
\indent \indent - míra ředění - $\alpha_{dil}$\\
\indent - míra degradace/ředění\\
\begin{equation}
\alpha=\alpha_{deg}+\alpha_{dil}
\end{equation}
- změna koncentrace Y v čase
\begin{equation}
dY/dt=\beta-\alpha Y
\end{equation}
- v ustáleném stavu, konstantní koncentrace $Y_{st}$, řešení  $dY=0$\\
\begin{equation}
Y_{st}=\beta/\alpha
\end{equation}
- odstranění signálu, konec produkce ... $\beta=0$\\
\begin{equation}
Y(t)=Y_{st}e^{-\alpha t}
\end{equation}
- \textbf{časová odezva} - měří rychlost změny Y, $T_{1/2}$\\
\indent - čas k dosažení poloviny počáteční a koncovou úrovní v dynamické procesu\\
\indent - z předchozí rovnice z $Y_{st}$ do $Y=0$ $\rightarrow$ $Y(t)=Y_{st}/2$\\
\begin{equation}
T_{1/2}=log(2)/\alpha
\end{equation}
- některé proteiny mají vysokou degradační míru $\alpha$,je tedy nutná vysoká produkce $\beta$\\
\indent - cyklus produkce a destrukce - rychlá reakce, když je potřeba změny\\
- $Y=0$, spuštění signálu, začátek hromadění $Y$, růst k $Y_{st}=\beta/\alpha$\\
\begin{equation}
Y(t)=Y_{st}(1-e^{-\alpha t})
\end{equation}

\subsection{Časová odezva stabilních proteinů je jedna generace buněk}
- stabilní proteiny - $\alpha_{deg}=0$, nejsou aktivně degradovány v rostoucích buňkách\\
\indent - produkce hlídána ředěním, $\alpha=\alpha_{dil}$\\
- produkce proteinu, náhlý konec ($\beta=0$), růst buňky do dvojnásobku, rozdělění ve dví\\
\indent po \textbf{jednogeneračním čase} $\tau$ - koncentrace klesá na 50\%\\
\begin{equation}
T_{1/2}=log(2)/\alpha_{dil}=\tau
\end{equation}
- tato časová odezva může být limitujícím faktorem, který představuje omezení pro tvorbu efektivních genetických obvodů\\


\chapter{Autoregulace: síťové motivy}
- cíle \\
\indent - \textbf{síťový motiv} - definovat vzor stavebních bloků v síti\\
\indent - zkoumat nejjednodušší motiv v transkipční síti - \textbf{negativní autoregulace}\\
\indent - ukázat, že tento motiv má užitečnou funkci - urychlení odezvy transk.interakce a stabilizace\\

\section{Vzory, náhodně vytvořené sítě, síťové motivy}
- E.coli - mnoho vzorů nodů a hran, hledání důležitých vzorů - statistika\\
\indent - porovnávání se souborem \textbf{náhodných sítí} \\
\indent \indent - sítě se stejným počtem nodů a hran jako reálné, náhodné spojení mezi nody a hranami\\
- \textbf{síťové motivy} - vzory opakující se výrazně častěji než v náhodných sítích\\ 
\indent - motivy odolávající mutacím, které náhodně upravují hrany\\

\subsection{Detekce motivů srovnáváním s náhodnými sítěmi}
- ER model - Erdos and Renyi - nejjednodušší soubor náhodých sítí\\
- pro hodnotné srovnání je třeba náhodných sítí se základními rysy reálných sítí\\
\indent - stejný počet hran (edge - E) a nodů (N) jako reálná, náhodně vytvořené spoje mezi nody\\
\indent - N(N-1)/2 možných párů nodů, možné oba směry $\rightarrow$ N(N-1)\\
\indent - hrana může do stejného nodu, ze kterého vycházela - N\\
\indent počet možných hran:
\begin{equation}
N(N-1)+N=N^2
\end{equation}
\indent - pravděpodobnost hrany $p=E/N^2$

\section{Autoregulace: motiv sítě}
- srovnávání transripční sítě E.coli s náhodnými\\
- \textbf{vlastní hrany} - vrací se do stejného nodu, ze kterého vyšly - 40\\
\indent - transkr.fak. regulující transkripci vlastního genu\\
\indent \indent - autogenní kontrola, \textbf{autoregulace}\\
\indent - 34 jsou zde represory - potlačení vlastní transripce - \textbf{negativní autoregulace}\\
\indent - statisticky by v náhodně vytvořené síti měla být pouze 1$\pm$1 vlastní hrana\\
\indent - vypočtená odpovídající odchylka Z $\sim$ 32 je statisticky velmi významná\\
\indent - vlastní hrany - autoregulace - je motiv sítě\\

\subsection{Urychlení odevzy obvodu genu negativní autoregulací}
- transrk.fak X se naváže na vlastní promotor k potlačení produkce mRNA\\
- čím vyšší koncentrace, tím nižší míra produkce\\
- dynamika popsána mírou produkce $f(X)$ a mírou degradace/ředění - pro aproximaci využit Hill\
- X na začátku není, t=0\\
\indent - rychlá produkce, dosažení vysokého ustáleného stavu, konec produkce na X=K (koef. represe)\\
\begin{equation}
T_{1/2}=\frac{K}{2\beta}
\end{equation}
\indent - čím silnější nerepresovaná aktivita $\beta$, tím rychlejší odezva\\
\indent - negativní autoregulací rychlé nabytí počáteční produkce, autorepresí ukončení produkce\\
\indent - evolucí jednoduché nastavení parametrů $\beta$ a K nezávisle na sobě\\
\indent \indent - K - například mutací v místě vázání X v promotoru\\
\indent \indent - $\beta$ - mutacemi v místě vázání RNAp v promotoru\\
\indent - dosažení výrazně vyšší rychlosti odezvy než v případě bez negativní autoregulace\\
- negativní autoregulace\\
\indent - silný promotor dá rychlou produkci\\
\indent - vhodný koeficient represe poskytuje žádoucí ustáleného stav\\
\indent - v jednoduché regulaci dosažení značně vyššího ustáleného stavu, potom nežádoucí nadměrná\\
\indent exprese produktu genu

\subsection{Negativní autoregulace podporuje robusnost v kolísání míry produkce}
- další výhodou negativní autoregulace zvýšení robusnosti expresní úrovně ustáleného stavu ke kolísání produkční míry $\beta$\\
- kolísání metabolické kapacity v buňce, regulačního systému, náhodné efekty v produkci\\
\indent $\rightarrow$ buněčná dvojčata rozdílné míry produkce $\beta$ u většiny proteinů\\
\indent - rozdíly i po dobu celé generace\\
\indent - represní práh K se většinou moc neliší\\
- lineární závislost regulace na kolísání produkční míry $\beta$ - $X_{st}=\beta/\alpha$
- u negativní autoregulace závisí hodnota ustáleného stavu pouze na represním práhu K - $X_{st}=K$\\
\indent - K se v porovnání buňka-buňka tolik neliší $\rightarrow$ zvýšení robusnosti ustáleného stavu z hlediska\\
\indent kolísání míry produkce\\
 
\subsection{Pozitivní regulace zpomaluje odezvu a může vést ke dvojí stabilitě}
- aktivace vlastní transkripce, 10\% transkr.fak. v E.coli, pomalá dynamika\\
- s růstem úrovně X roste míra produkce\\
- dosahuje polovičního zpoždění ve srovnání s jednoduchou regulací\\
- vhodné pro procesy trvající dlouhou dobu - vývojové procesy\\
- dvojí stabilita - míra pozitivní autoregulace je silná v porovnání míry degradace/ředění\\
- při aktivaci genu zůstane aktivní i po zmizení vsupního signálu\\
- využití ve vývojových transripčních sítích, kde je potřeba stanovit osud buňky\\


\chapter{Motiv sítě dopředné smyčky}
- mnoho vzorů, jen ty nalézané významně jsou motivy\\
- motivy mají definovanou funkčnost zpracování informace\\
\indent - jeich benefity vysvětlují, proč evolucí znikají v různých organismech\\
- zde zaměření na motivy se 3 nody\\
\indent - 13 možností, pouze jedna je motiv sítě (feed-forvard loop FFL)\\
\indent - důležité pochopit regulace jednotlivých hran - každá může být buď aktivátor nebo represor\\
\indent - 8 FFL typů, poze 2 se vyskytují častěji v sítích\\
\indent - zde zabývání se dynamikou, běžné typy filtrují šum, pulsy a odezvové zrychlení\\

\section{Výskyt subgrafu v náhodných sítích}
- \textbf{subgraf} - vzor z více nodů\\
- \textbf{dopředná smyčka} - 3 nody, nevrací se do počátečního\\
- výskyt trojúhelníkových subgrafů je náhodných sítích velice vzácný\\
- při výskytu v reálné síti lze považovat za motiv\\

\section{Dopředná smyčka je motiv sítě}
- v E.coli 42 dopředných smyček, žádná se zpětnou vazbou (3 nody spojeny dokola, často jako anti-motiv)\\
\indent - FFL je silný motiv, mnohem častěji než v náhodných\\
\indent - FFL jediný motiv ze všech 13 3-nodových možností\\

\section{Struktura dopředné smyčky genetického obvodu}
- transkr.fak. X reguluje traskr.fak. Y a oba regulují transkr.fak. Z, 2 paralelní regulační cesty, 8 typů\\ 
\indent - \textbf{koherentní} - nepřímá cesta má stejné celkové znaménko jako přímá cesta\\
\indent - \textbf{nekoherentní} - nepřímá cesta má celkově opačné znaménko co přímá\\
- nejčastěji C1-FFL (coherent 1) - všechny aktivátory - +/++\\
- 2. nejčastější je I1-FFL (incoherent 1) - +/+-\\
- všechny ostatní se objevují výrazně méně často\\
- integrace X a Y do promotoru Z - aktivace - AND/OR\\
- X a Y často reagují na vnější podněty - např. molekuly vázající se na transkr.fak.
- podnět z venku většinou zpracováván výrazně rychleji než vnitřní transkripční interakce v FFL\\
- příchod $S_x$, aktivace $X$ $\rightarrow$ $X^*$, navázaní na dané místo DNA promotoru genu Y a Z během pár $s$,\\
\indent také změna míru transkripce $\rightarrow$ koncentrace $Z$ se mění v intervalu od minut do hodin\\

\section{Dynamika C1-FFL s logikou AND}
- signál $S_x$, aktivace na $X^*$ (\textbf{kroková stimulace} X), navázání k promotoru $Y$, produkce proteinu $Y$
\indent $\rightarrow$ druhý transrk.fak. v FFL\\
- paralelně se $X^*$ váže na promotor genu $Z$ - logika AND - nestačí k aktivaci\\
\indent - aktivace po překročení kocentrace $Y$ \textbf{aktivačního práhu} genu $Z$ - $K_{yz}$\\
\indent \indent - aktivace vyžaduje i 2. vstupní signál $S_y$ - Y v aktivní formě $Y^*$\\
- příchod $S_x$, čekání na akumulaci Y pro aktivaci Z - \textbf{zpoždění}\\

\section{C1-FFL je prvek zpoždění citlivý na znaménka}
- zpoždění s ON krokem $S_x$, ne při OFF kroku - zpoždění citlivé na znaménko\\
- může být bráno jako druh asymetrického filtru\\
\indent - signál $S_x$ s kratší dobou trvání než odezva nijak dál nepůsobí na Z - \textbf{detektor trvání} na ON\\
\indent - okamžitá odezva na OFF pulsy\\
- inženýrské použití - když náklady na chybu jsou nesymetrické\\
- transkripční sítě - spíše ochranná funkce, filtrace kolísání\\
- případ OR - odpadá zpoždění při ON, vzniká při OFF - buzení Z i když zrovna vypadnul signál\\
\indent - zpoždění cca o hodinu\\

\section{I1-FFL}
- X aktivátorem Y i Z, Y represorem Z\\
- $S_x$ $\rightarrow$ aktivace na $X^*$, vazba k Z, inicializace transkripce - produkce proteinu\\
\indent - paralelně aktivace produkce Y, nahromadění nad práh, tlumení Z\\
- I1-FFl tedy může generovat pulsy produkce Z\\
- urychlení odezvy - nejdříve vysoká produkční míra, se zpožděním represor na snížení produkční míry\\
\indent - udržení žádoucí úrovně ustáleného stavu\\
- při zmizení signálu $S_x$ žádná akcelerace nebo zpoždění - okamžité ukončení produkce Z\\
\indent - kvůli AND logice promotoru Z\\
\indent \indent - s logikou OR funguje stejně, akorát je odezva při OFF a ne při ON\\
\indent - po ukončení exponenciální pokles Z podle míry degradace/ředění\\
\indent - I1-FFL je prvek zpoždění citlivý na znaménka\\

\section{Proč jsou jiné FFL typy tak vzácné?}
- některé postrádají citlivost k jednomu z jejich dvou vstupů\\
- nepřítomnost signálu $S_y$ u I1-FFL - Z není represován, velký vliv na úroveň ustáleného stavu Z\\
- I4-FFL - +/-+ - cca 5\% z FFL\\
\indent - urychlovač s citlivostí na znaménko i pulsní generátor\\
\indent - narozdíl od I1-FLL nezávisí na $S_y$, při nepřítomnosti $S_x$ je nesplněna logika AND z X do Z\\
\indent \indent - nepřítomnost citlivosti vůči jednomu vstupu je důvod, proč se nevyskytuje moc často\\
\indent \indent \indent - stejný důvod i pro I3-FFL\\
\indent \indent \indent - podobný důvod i u C3-FFL a C4-FFL\\
\indent \indent \indent - u ostatních případů a OR formy je složitější vysvětlování\\

\section{Konvergentní evoluce FFLs}
- základní V-tvar kdy X a Y regulují Z, od jedné do pár mutací se může přidat i regulace X $\rightarrow$ Y\\
\indent - pokud nepomůže nebo ještě zhorší - rychle evolucí odstraněna
\indent - u C1-FFL a I1-FFL je fukce jasná\\
- \textbf{homologní geny} - geny z jednoho společného, značný stupeň podobnosti\\
\indent - možný podobný vznik u FFLs - většinou však ne\\
- vznik spíše nezávislý na stejných obvodech ve spoustě případech


\chapter{Časové programy a globální struktura transkripčních sítí}
- zde kompletace průzkumu smyslových motivů sítě - předtím autoregulace a dopředné smyčky\\
- další 2 motivy\\
\indent - SIM - single-input module\\
\indent \indent - 1 regulátor kontroluje skupinu genů\\
\indent \indent - vytváří časové programy exprese - geny jsou zapnuty postupně jeden po druhém ve\\
\indent \indent \indent stanoveném pořadí\\
\indent \indent - "just-when-needed" strategy - není produkce proteinu, dokud neni potřeba\\
\indent \indent - optimální pro rychlou produkci systému, který se skládá z různých typů proteinů\\
\indent \indent \indent - s omezeným zdrojem pro produkci těchto proteinů\\
\indent - DORs - dense overlapping regulons - hustě se překrývající regulony\\
\indent \indent - husté pole regulátorů kombinatoricky kontrolující výstupní geny\\
\indent \indent - můžou nést výpočty důležité pro rozhodování \\
- nakonec, jak motivy zapadají do sebe, aby vytvořily síť

\section{Síťový motiv SIM}
- větší motiv - více vzorů pohromadě s typickým vzorem architektury - 1. rodina motivů je SIM\\
- hlavní transrk.fak. X kontroluje skupinu cílových genů - každý jeden vstup\\
\indent - znaménko (aktivace/represe) pro všechny kontrolované nody stejné\\
- X bývá často autoregulované\\
- silný motiv oproti náhodným sítím\\
- geny mají běžné biologické funkce - geny se účastní v nějaké metabolické cestě\\
- geny pracují sekvenčně pro nashromáždění žádoucí molekuly atom po atomu\\
- jiné  kontrolují geny reagující na stresové podněty - poškození DNA, teplotní šok...\\
\indent - produkce proteinů opravujících škody\\
\indent - podskupiny genů specializovaných na dané aspekty\\

\section{SIM umí generovat časové programy exprese}
- aktivace genů v daném pořadí\\
- různé práhy X pro každý cílový gen\\
\indent - práh pro každý promotor závisí na místě jeho vázání s X v jeho promotoru\\
- postupná aktivacce genů, při úbytku zas postupná deaktivace z druhé strany - LIFO\\
- v E.Coli systém argininu, SOS DNA poničení - zpoždění v pořadí o 0.1 generaci genu (5-10 min)\\
- ekonomický design - produkce až ve chvíli potřeby\\
- přesné \textbf{časový řád} může být pozměněn mutacemi\\ 
\indent - při opravách rychlá produkce opravných prostředků, po ukončení různě rychlý pokles\\
\indent \indent - geny řešící nejmenší problémy se vypnou první, ty pro rozsáhlejší se vypínají později\\
\indent - řeší celou řadu globálních buněčných odezev - geny časované podle buněčného cyklu v bakterii\\
\indent - geny často regulované hlavním regulátorem + přídavnými regulátory pro subsystémy\\
\indent \indent - nemusí být tedy SIM\\
\indent - více regulátorů, jeden má konstatní aktivitu během sledovaného intervalu\\
- vznik SIM evoluční konvergencí ke stejnému regulačnímu vzoru u různých organismů - jako u FFL\\
- udžuje se vůči mutacím, protože je dostatečně užitečný\\
- někdy proces aktivace-deaktivace jako FIFO\\
\indent - některé části procesu potřebné dříve jiné později, ale jen jednou\\

\section{Topologická generalizace síťových motivů}
- zatím jednoduché - už 4 nody 199 možných vzorů, přes 9000 pro 5\\
- \textbf{topologická generalizace motivů} - rozdělení motivů do rodin dle jejich funkcionality\\
\indent - když je FFL motiv, každý z těch 3 vzorů může být motiv - pouze 1 je ale opravdu motiv\\
\indent \indent - \textbf{multi-output FFL} - může generovat časový FIFO program\\

\section{multi-output FFL umí generovat časové FIFO pořadí}
- genetický systém kontrolující produkci flagel (bičík) - mototry E.coli\\
- když je buňce fajn, nikam se nepřemisťuje, když špatně nechá si narůst motůrky, vygeneruje navigační systém\\
- bičík cca 50nm, z 30 typů proteinů, 10x delší než tělo, 30 mikronů/s\\
\indent - skládá se dohromady po fázích - lego\\
\indent \indent - uvnitř kanálek, kterým se další proteiny dostávají dál\\
\indent - informace o proteinech kódovaná v 6 operonech\\
\indent \indent - skupina genů transkribována stejným kusem mRNA\\
\indent \indent - operony regulovány 2 transkr.fak. - aktivátory\\
\indent \indent \indent - hlavní aktivátor X aktivuje Y, oba aktivují každý ze 6 operonů\\
\indent \indent \indent \indent - \textbf{multi-output FFL}\\
\indent \indent \indent - každý operon může být aktivován pouze pomocí X nebo Y - podobné OR branám\\
\indent \indent \indent - just-when-needed produkce\\
\indent \indent \indent - + oproti SIM - FIFO - X najede, postupně překročí hranice aktivace promotorů\\
\indent \indent \indent \indent - klesáním zpět by bylo opět LIFO - X klesne, ale je furt Y\\
\indent \indent \indent \indent - Y má vlastní hranice pro každý gen - jiné pořadí na OFF než na ON\\
\indent \indent \indent - dále všechny funkce FFL - vytváření zpoždění při vypínání Z \\
\indent \indent \indent \indent - deaktivace po určitém čase - detektor zpoždění pro každý výstup\\
\indent \indent \indent - při krátké ztrátě X se díky OR nemusí přerušit vstup\\
\indent \indent \indent - multi-output je generalizace FFL, která se vyskytuje nejčastěji v transkr. sítích\\

\section{Integrace signálu a kombinatorická kontrola: bi-fan a hustě se překrývající regulony}
- 4-nodové vzory - ze 199 možností jsou známé 2 motivy - two-output FFL a \textbf{bi-fan}\\
\indent bi-fan - $X_1$ a $X_2$ oba regulují $Z_1$ $Z_2$\\
\indent \indent - patří do skupiny \textbf{hustě se překrývajících regulonů - DORs}\\
- DOR - řada transkr.fak. regulující sadu výstupních genů v hustě se překrývající směru\\
\indent - většinou ne všechny vstupy regulují všechny výstupy\\
\indent - kombinatorický rozhodovací systém\\
\indent - E.coli a kvasnice pár DORs - regulace někalika set genů\\
\indent - sdílená společná globální funkce - reakce na stres, metabolismus živin...\\

\section{Síťové motivy a globální struktury smyslových transkripčních sítí}
- autoregulace, FFL, SIM, DOR - jak jsou vůči sobě, jak se překrývají?\\
- procedura hrubého zrna - substituce motivů za tvary - zjednodušení grafického znázornění sítě\\
- všechny geny jsou pokryty nějakým z rodiny motivů\\
- poměrně vzácné dlouhé kaskády ve smyslových transkr.sítích - časové nároky\\
\indent - jiné biologické sítě je často obsahují\\


\chapter{Síťové motivy ve vývojových, transdukčních a neuronových sítích}
- vše předtím ve smyslových sítích - potřeba rychlých reakcí\\
- byly tvořeny malými sadami těchto motivů\\
- tady - síťové moivy v jiných typech bio sítí\\
- \textbf{vývojová transkripční síť} - transkr. sítě řídící osudy buněk - z vajíčka do mnohobuněčného org\\
\indent - diference buňky na jiný typ buňky\\
\indent - od smyslových rozdíl v časových intervalech a reverzibilitě\\
\indent \indent - pomalejší nevratná rozhodování - vznik nových motivů sítě\\
- kromě trankr. sítí používá buňka další interakčí sítě\\
\indent - interakční síť protein-protein, signálová transdukční síť, metabolické sítě\\
\indent \indent - graficky různé barvy hran pro rozeznání různých sítí\\
\indent \indent - rozdíly v časových intervalech\\
\indent \indent \indent - transkr.sítě v hodinách, signálová transdukční síť funguje v sekundách až minutách\\

\section{Síťové motivy ve vývojových transkripčních sítích}
- \textbf{senzorová transkr.síť} - reakce na vnější změny, skoro ve všech buňkách\\
- \textbf{vývojové transkripční sítě} - diferenciační procesy, vývoj, vajíčko $\rightarrow$ vícebuněčný organismus\\
\indent - rozdělením diferencují do jiné tkáně/pletiva\\
\indent \indent - aby se staly součástí nové tkáně, musí vyjádřit specifickou sadu proteinů\\
\indent \indent \indent - sada určuje, jestli bude tkáň svalová nebo neurální\\
\indent - zkoumání na octomilkách, červech, mořských ježcích a lidech - pár silných motivů\\
\indent \indent - C1-FFL a I1-FFL jako u senzorových sítí, taky autoregulační motivy a SIMs\\
\indent \indent - dále pak jiné, co u senzorových sítí nebyly\\

\subsection{Dvounodová pozitivní zpětná smyška pro dělání rozhodování}
- dva transkr.fak. regulují samy sebe - ve vývojových sítích spíše pozitivní\\
\indent - 2 pozitivní interakce - double-positive feedback - transkr.fak. se navzájem aktivují\\
\indent \indent - 2 ustálené stavy - X a Y oba ON / oba OFF\\
\indent \indent - signál pro produkci X/Y může nereverzivně uzamknout oba proteiny do ON\\
\indent \indent \indent - vzájemná aktivace\\
\indent \indent \indent - \textbf{blokovací mechanismus}\\
\indent \indent - neujúčinější, když geny regulované X a Y kódují proteiny patřící stejné tkáni\\
\indent - 2 negativní interakce - double-negatice feedback - vzájemná represe\\
\indent \indent - také 2 stabilní ustálené stavy - X ON a Y OFF / X OFF a Y ON\\
\indent \indent \indent - vyjádření buď X nebo Y \\
\indent \indent - vhodné, když geny regulované X patří buňkám s jiným osudem než geny regulované Y\\
- často ještě u obou nodů pozitivní autoregulace\\
\indent - zvyšování produkce při dosažení určité hranice\\
\indent \indent - tím také stabilizace ON ustálený stav transkr.faktoru\\
- bistabilní přirozenost těchto motivů dovoluje buňkám dělat nereverzivní rozhodnutí a přiřadit osudy, kdy specifická sada genů se vyjadřuje a další je utlumená\\

\subsection{Regulující zpětná vazba a regulovaná zpětná vazba}
- 2 hlavní 3-nodové motivy - obsahují zpětnou vazbu\\
\indent - trojúhelníhový motiv, kde X a Y vzájemně regulují Z - regulující zpětná vazba
\indent \indent - 10 možných kombinací znamének\\
\indent \indent \indent -  double-positive - stejné znaménko z obou do Z\\
\indent \indent \indent -  double-negative - často rozdílné znaménko do Z\\
\indent \indent - také 2-nodová smyčka, kde jsou oba nody společně regulovány zezhora 1 transkr.fak.\\ 
\indent \indent \indent - regulovaná zpětná vazba\\
\indent \indent \indent - prvek paměti - Z může řídit zapnutí/vypnutí smyčky\\
\indent \indent \indent \indent - smyčka si pamatuje, jestli bylo Z aktivní nebo ne\\
\indent \indent \indent \indent - zapamatování si osudu buněk, i když už zmizel původní signál\\

\subsection{Dlouhé transkripční kaskády a vývojové časování}
- dlouhé transkripční kaskády vzácné v senzorových sítích, ve vývojových motivy\\
- řetězy interakcí - X $\rightarrow$ Y $\rightarrow$ Z $\rightarrow$ ...\\
- čas odezvy každého článku  je daný mírou degradace/ředění v $T_{1/2}$ - odezva kolem jedné generace\\
\indent - u vývojových sítí tato odezva sedí líp - lepší řízení vývojových procesů\\

\subsection{Propojená dopředná smyčka v B.subtilis sporulační sítí}
- sporulace - proces tvorby spor - buňka sloužící k dlouhodobému přežití bakterie v nepříznivých podmínkách\\ 
- b.subtilis - bacil senný\\
- ve vývojových sítích FFL často jen část větších a složitějších obvodů, než u senzor.sítí\\
- při hladovění se buňky přestanou dělit a diferencuje do odolných sporů\\
\indent - spory obsahují mnoho proteinů, co se v bakterii při růstu nevyskytují\\
\indent - odpočívající buňka, skoro úplně dehydrovaná - přežije dlouho, spící stav\\
\indent - při správných podmínkách se tranformuje zpět na normální bakterii\\
- při tvorbě sporu je třeba změnit tvorbu sady proteinů - sporulace\\
\indent - potřeba stovek genů - spínání ON a OFF ve vlnách - každá specifický vliv na formaci sporu\\
\indent - síť tvořena několika transkr.fak. sežazenými ve spojeném C1 a I1-FFL\\
\indent - nakomibování FFLs pro využití jejich zpoždění a generování pulsů - časový program genetické\\
\indent \indent exprese\\
\indent - kaskádově sežazené FFLs generují 2 pulsy genů následované třetí pozdním\\

\section{Síťové motivy v signálových transdukčních sítích}
- \textbf{signálová transdukční síť} - mnohem rychlejší zpracovávání informací než 1 generace\\
- interake mezi signálovými proteiny\\
- vycítit informaci z okolí, zpracovat, regulovat aktivitu transkr.fak\\
- vstupy obvykle detekovány \textbf{receptorovými proteiny}\\
\indent - jeden konec mimo buňku, druhý uvnitř v cytoplasmě\\
\indent \indent - mimobuněčný konec detekuje molekuly zvané \textbf{ligandy}\\
\indent \indent \indent - navázáním konformační změna v receptoru, vnitří část se stane aktivní a katalyzuje\\ \indent \indent \indent \indent dané chemické modifikace do difusního messenger proteinu v buňce\\
\indent \indent \indent \indent - poslání 1 bitu informace z receptoru do messengeru\\
\indent \indent \indent - modifikace funguje v rámci sekund až minut\\

\section{Informace zpracovávaná pomocí vícevrstvých perceptronů}
- v signálových transdukčních sítích jsou nody signálové proteiny a hrany interakce\\
- silné 2 4-nodové motivy - bi-fan, diamant\\
- diamant - vícevrstvý vzor, podobný strukturám DOR v kaskádě s DOR, který dostává informaci z nadřazeného DORu\\
\indent - vzory DOR se ale normálně nevskytují v kaskádě\\
- \textbf{vícevrstvé perceptrony} - podobná struktura využívaná v umělé inteligenci a neuronových sítích\\

\subsection{Tréninkový model pro perceptrony proteinové kinázy}
- \textbf{kináza} - enzym, který přenáší fosfátovou skupinu z vysokoenergetické donorové molekuly (např. ATP) na určitou cílovou molekulu (substrát)\\
- \textbf{kaskáda proteinové kinázy} - cesty zpracování informací nalezené ve většině eukaryotických organismů\\
- aktivace kaskády, když se receptor naváže na ligand a aktivuje první kinázu X $\rightarrow$ \\
\indent $\rightarrow$ X fosforylace kinázy Y na dvou daných místech $\rightarrow$ Y dvojně zfosforylovaná $\rightarrow$ \\ \indent $\rightarrow$ začne fosforylovat Z $\rightarrow$ po dvojné fosforylaci fosforyluje transkr.fak. $\rightarrow$ exprese genu\\
- fosfatázy - enzymy proteinů postupně defosforylující kinázy - také obsažené v kaskádách\\
- často využito lešenářské proteiny držící kinázy u sebe\\
- adaptorové proteiny můžou spojit danou kaskádu k rozdílným vstupním receptorům v odlišném typu buňky\\
\indent - kaskády tedy znovuvyužitelné moduly\\ 
\indent - stejná kaskáda transdukuje rozdílný signál v rozdílné tkáni\\
- kaskády většinou ve vrstvách, často 3\\
\indent - v 1. řadě kinázy $X_1$, $X_2$... aktivují další řadu kináz $Y_1$, $Y_2$... ...\\
\indent \indent - formování vícevrstvého perceptronu který může integrovat vstupy z více receptorů\\
- \textbf{kinetika prvního řádu} - nejednodušší kinetika pro kinázy\\
\indent - míra fosforylace Y z X je proporční vůči koncentraci aktivního X násobené koncentrací jeho\\
\indent \indent substrátu - nefosforylované Y, značení $Y_0$\\
\begin{equation}
rate_.of_.phosphorylation=vXY_0
\end{equation}
\indent míra $v$ kinázy X
- kináza Y fosforylovaná 2 rozdílnými vstupními kinázami - $X_1$ $X_2$
\indent - fosforylovaná forma $Y_p$, nefosforylovaná forma $Y_o$\\
\begin{equation}
Y=Y_o+Y_p
\end{equation}
\begin{equation}
dY_p/dt=v_1X_1Y_o+v_2X_2Y_o+\alpha Y_p
\end{equation}
- stabilní stav - $dY_p/dt=0$\\
- řešení\\
\begin{equation}
Y_p/Y=f(w_1X_1+w_2X_2)
\end{equation}
\indent f - rostoucí saturující funkce\\
\begin{equation}
f(u)=\frac{u}{1+u}
\end{equation}
\indent w - váhy vstupů\\
\begin{equation}
w_1=v_1/\alpha
\end{equation}
- koncentrace fosforylovaného Y je rostoucí funkce vážených sum dvou aktivit kinázy\\
- když je Y kináza, která musí být fosforylována na dvou místech aby byla aktivní, vstupní funkce je strmější\\
\begin{equation}
f(u)=\frac{u^2}{1+u+u^2}
\end{equation}
- S-shaped vstupní funkce vedou k aktivaci Y pouze pokud je suma vstpů větší než mezní hranice - tady 1\\
- \textbf{hranice aktivace} - $w_1X_1+w_2X_2=1$\\
\indent - přímka v ploše vymezené vstupními aktivitami

\subsection{Vícevrstvé perceptronu mohou provést detailní výpočty}
- jednovstvý perceptron $X_1$-$X_2$ rozdělí plochu přímkou do 2 částí - přidáním vstvy možnost složitějších výpočtů\\
- 2 vstvy - $Y_1$ a $Y_2$ mají vlastní sady mezí ze 2 vstupů $X_1$, $X_2$ - rozdělení v půlce\\
\indent - oblasti nízké a vysoké aktivity\\
- kinázy $Y_1$ a $Y_2$ můžou fosforylovat kinázu Z - pouze pokud jsou fosforylované\\
\begin{equation}
Z_p/Z=f(w_{z1}Y_1+w_{z2}Y_2)
\end{equation}
- když jsou váhy $w$ malé, Y musí být fosforylované pro překročení hranice\\
\indent - Z je fosforylované pouze na ploše $X_1$-$X_2$, když jsou Yny aktivní\\
\indent - oblast dána průsečíky dvou regionů aktivity $Y_1$ a $Y_2$\\
\indent \indent - aktivace Z v regionu vyhrazeném 2 úsečkami\\
- když v prostřední vrstvě místo kinázy specifická fosfatáza - fosfatáza odstraní fosforylní úpravu - negativní váha\\
\indent - složitější aktivační regiony\\
- \textbf{diskriminace} - schopnost rozeznat jistý stimulující vzor\\
\indent - nastavením vah možnost rozeznat od sebe velmi podobné stimulující vzory\\
- \textbf{generalizace} - "vyplňování mezer" v částečných stimulujících vzorech\\
\indent - vpuštěním neúplného vzoru obdov reaguje jako by byl celý, pokud je mu daná část podobná\\
- \textbf{pozvolná degradace} - vadou na prvcích sítě nedojde k úplnému ukončení\\
\indent - výkon se zhoršuje úměrně poškození\\
- tyto 3 fenomény můžou charakterizovat funkcionalitu signálové transdukční sítě v buňce\\

\section{Kompozitní síťové motivy: záporná zpětná vazba a oscilátorové motivy}
- zatím proteinové signálové sítě a transkripční sítě zvlášť - v buňce ale pracují pospolu\\
\indent - často výstup signáloé transduční cesty je transrk.fak\\
- spojená síť se 2 barvami hran - transripční a protein-protein interakce\\
- motivy\\
\indent - 3 nody - trnaskr.fak. X transkripčně reguluje 2 geny Y a Z, protein.prod. pak přímo interagují\\
\indent \indent - např. Y fosforyluje Z\\
\indent - kompozitní zpětnovazební smyčka ze 2 proteinů, které spolu interagují různými barvami\\
\indent \indent - X transkr.fak aktivující gen Y, produkt Y interaguje s X na proteinové úrovni\\
\indent \indent \indent - často opačným způsobem\\
\indent \indent \indent - Y naváže na X a potlačuje jeho aktivitu transkr.faktoru bráněním přístupu k DNA\\
\indent \indent \indent \indent - nejčastěji se objevuje v genetickém systému bakterie-člověk\\
\indent \indent \indent - pomalá aktivace Y, rychlé potlačování X - negativní stabilizace\\
\indent \indent \indent \indent - stabilnější, než 2 pomalé interakce\\
- homeostáza - stabilita okolo fixovaného stavu - v mnoha systémech žádoucí\\
- oscilační dynamika - chování u některých systémů\\
\indent - cyklus buňky, kdy se periodicky duplikují genomy\\
\indent - cirkadiánské hodiny - pozoruhodně přesné biochemické obvody produkující oscilace na škále\\
\indent \indent jednoho dne\\
\indent ...\\
\indent - typický charakter - časování výrazně přesnější než amplituda\\
\indent \indent - variace amplitudy způsobená měnícím se vnitřním šumem z produkce proteinů\\
\indent - mnoho oscilátorů bývá implementováno pomocí 2-barevného motivu kompozitní negativní\\
\indent \indent zpětnovazební smyčky\\
\indent \indent - transkr.fak. X má pozitivní autoregulaci\\
\indent \indent \indent - robustní časování i přes fluktuace biochemických parametrů komponenty\\
\indent \indent \indent - rodina relaxačních oscilátorů\\
\indent - zpětnovazebné smyčka složená z regulátorů poskládaných dokola tvořící negativní zpětnovazební\\
\indent \indent smyčku\\
\indent \indent - 3 represory za sebou - represilátor\\
\indent \indent \indent - rodina oscilátorů zpoždění - méně precizní časování\\

\section{Síťové motivy v neuronové síti C.elegans}
- Ceanorhabditis elegans - háďátko obecné - půdní červ - cca 1000 buněk\\
\indent - nesouvisející sítě mají shodné motivy, anti-motivy a častěji se opakující různé vzory\\
- nody - neurony\\
- $X\rightarrow Y$ - X má synaptický spoj s Y\\
- podobné motivy jako v biochemických interakčních sítích - i přes odlišné prostorové a časové vazby\\
\indent - komunikace mezi buňkami v rámci milisekund\\
\indent - v transkr.síti v rámci buňky v časoech od minut po hodiny\\
- FFL - sprostředkovává přenos zašuměných signálů pomocí komponent šumu\\

\subsection{Vícevstupové FFL v neuronových sítích}
- v obou motivy, ale FFL jsou zde spojeny jinak než v transkr.sítích\\
\indent - v transkr.sítích generalizací multi-output FFL nejčastější\\
\indent - v neuronových u C.elegans multi-input\\
- zjednosušeně model aktivity neuronů má rovnice podobné těm v transkr.sítích a signálových\\
\indent transdukčních sítích, molekulární mechanismy jsou ale velmi odlišné\\
- neurony komunikují přenosem elektrických signálů přes sinapse do dalších neuronů\\
- neurony mají časově závislý transmembránový rozdíl napětí - aktivita neuronu\\
- u c.elegans odstupňované napětí - X(t), Y(t), Z(t)\\
- integrate-and-fire - model pro dynamiku neuronů - sečtení synaptických vstupů ze vstupního neuronu\\
- Y má 2 synaptické vstupy ze 2 neronů $X_1$ a $X_2$, změna napětí Y aktivována skokovou funkcí přes váhovou sumu napětí dvou vstupních neuronů\\
\begin{equation}
dY/dt=\beta\theta(w_1X_1+w_2X_2>K_y)-\alpha Y
\end{equation}
\indent $\alpha$ - relaxační míra vůči úniku proudu přes membránu neuronové buňky\\
\indent váhy w - síla synaptických spojení
\indent - AND i OR brány\\
- \textbf{detekce náhody} krátkých vstupních signálů - aktivace Z i když nebyly přítomné oba vstupní\\
\indent signály, ale oba jen na krátkou chvíli a chvíli po sobě\\
- časová odezva v desítkách milisekund\\

\subsection{Vícevrstvé perceptrony v neuronové sítí C.elegans}
- podobné jako v signálových transdukčních sítích, zde větší hojnost vzájemných spojení\\
- zpracovávání informací, správnost závisí na přesnosti měření vah hran\\


\chapter{Robusnost proteinových obvodů: příklad bakteriální chemotaxe}
 - chemotaxe - pohyb organismu ve směru chemického gradientu - bílá krvinka k zánětu\\

\section{Princip robustnosti}
- výpočty v biologických obvodech závisí na biochemických parametrech - koncentrace proteinu\\
- paramtery odlišné buňka od buňky - náhodné jevy\\
- biologické obvody mají robustní konstrukci tak, že jejich základní funkce je téměř nezávislá na biochemických parametrech, které mají tendenci se měnit od buňky k buňce - \textbf{robustnost}\\
\indent - musí se uvést jaká vlastnost je robustní k jakému parametru\\
- fine-tuned - nerobustní vlastnosti\\
\indent - značně se mění při odlišných biochemických parametrech\\
- robustnost vůči změám parametru není nikdy absolutní\\
- demonstrace principů robustnosti na proteinových signálových sítích - obvod kontrolující bekteriální chemotaxi\\

\section{Bakteriální chemotaxe, jak bakterie myslí}
\subsection{Chování chemotaxe}
- pipeta s živinami k plujícím E.coli, jsou zaujaty a vytvoří obláček kolem - \textbf{atraktant}\\
\indent - když se škodlivými, uplavou pryč - \textbf{repelent}\\
\indent - \textbf{bakteriální chemotaxe}\\
- některé buňky pohyb za světlem (fototaxe), magnetickým polem (magnetotaxe)\\
- baktrie dokáží detekovat koncentrační gradienty malé jako změna molekuly na objem buňky na mikron a funkce v pozadí koncentrací přes 5 řádů\\
\indent - i přes lomcování Brownovým pohybem - buňka chce rovně po dobu 10s - orientace náhodně v\\
\indent \indent rozsahu $90^o$\\
- \textbf{dočasné gradienty} - k řízení pohybu - využití biased-random-walk\\
\indent - když buňka pluje od gradienty, zastaví se rychleji, aby našla nový směr\\
\indent - v tekutině se pohybuje náhodnou procházkou\\
\indent - \textbf{běh} - udržování konstantního směru - 1s\\
\indent - \textbf{tumbles} - přemet - změna směru - 0.1s\\
\indent - při přbližování snižování pravděpodobnosti přemetu (\textbf{frerkvence přemetu})\\
\indent \indent pokračování stejným směrem\\
- běhy a přemety jsou generovány různými stavy motoruů, které otáčí bičíky\\
\indent - clockwise CW, counterclockwise CCW\\
\indent - CCW - pluje dopředu, jedna se změní na CW, konec pohybu, přemet, náhodná orientace\\
\indent \indent - motor na CCW - znova pohyb dopředu\\

\subsection{Odezva a přesná adaptace}
- pozorování buňky bez gradientu, běhy a přemety s průměrnou \textbf{ustálenou frekvencí přemetu}\\
\indent - cca 1s\\
- atraktant ve vzduchu nad tekutinou, zatím žádný prostorový gradient, snížení frekvence přemetů\\
- buňky zjistí, že byly napáleny, zvýšní frekvence přemetů, i když je atraktant stále na místě - \textbf{adaptace}\\
- \textbf{přesná adaptace} - frekvence přemetů zůstane stejná, když se objeví atraktant ve stejné míře jako předtím\\
\indent - ustálená frekvence přemetů je nezávislá na úrovni atraktantu\\

\section{Chemotaxinový proteinový obvod E.coli}
- proteinový obvod vykonávající odezvu a výpočty adaptace\\
\indent - vstup - koncentrace atraktantu\\
\indent - výstup - pravděpodobnost sepnutí motorů na CW - určení frekvence přemetu\\
- \textbf{receptory} - vnímání atraktorů a repelentů - 5 druhů\\
\indent - část venku, druhá vevnitř - navázání \textbf{ligandů}\\ 
\indent - uvnitř receptor navázaný na proteinovou kinázu (CheA) - receptor a kináza - dohromady X\\
\indent \indent - X rychle mezi 2 stavy - aktivní $X^*$/neaktivní - mikrosekundy\\
\indent - aktivní $X^*$ - změna na příslušný regulátor proteinu CheY, který je rozptýlenýv buňce\\
\indent \indent - změna je přídavek fysforylační skupiny ($PO_4$)k CheY pro formování fosfo-CheY\\
\indent \indent - značení CheY-P, \textbf{fosforylace} - poslání bitů informací mezi signálovými proteiny\\
\indent \indent - CheY-P naváže na motor bičíku, zvýší pravděpodobnost změny z CCW na CW\\
\indent \indent \indent - čím všší koncentrace CheY-P, tím vyšší frekvence přemetů\\
\indent \indent - fosforylace CheY-P je odstraněna enzymem CheZ\\
\indent \indent \indent - proti sobě působící CheZ a aktivní $X^*$ - ustálený stav CheY-P a frekvence přemetů\\

\subsection{Atraktanty snižují aktivitu X}
- navázání ligandu, změna pravděpodobnosti aktivního stavu $X^*$\\
- \textbf{aktivita X} - koncentrace X v aktivním stavu\\
- navázáním atraktantu pokles aktivity - redukce míry fosforylace CheY od X - tedy i poklesu CheY-P\\
\indent - pokles pravděpodobnosti rotace motoru CW, redukce frekvence přemetu, delší plavba 1 směrem\\
\indent - repelenty opačný efekt\\

\subsection{Adaptace je díky pomalé modifikaci X, která zvyšuje jeho aktivitu}
- obvod chemotaxe má druhou cestu věnovanou adaptaci\\
- \textbf{metylační modifkace} - receptory mají pár biochemických tlačítek, která při stlačení zvyšují aktivitu a potlačují útlum X\\
\indent - metylová skupina ($CH_3$) přidána na 4-5 míst receptoru\\
- metylace receptoru je katalyzována enzymem CheR a je odstraněna enzmem CheB - metylové skupiny jimi průběžně přidávány a ubírány\\
\indent - pouze ale pokud bakterie necítí ligand\\
\indent - vypadá zbytečně, ale dává buňce možnost adaptace\\
- výrazně pomalejší než hlavní reakce z X na CheY do motoru\\
- design zpětné smyčky je takový, že je dosaženio přesné adaptace\\
\indent - zvyšující se metylace X přesně vyvažuje redukci aktivity, která je způsobena atraktantem\\

\section{Dva modely můžou vysvětlit přesnou adaptaci: robustní a fine-tuned}
- zjednodušené modely zanedbávající mnoho detailů, cíl pochopení základní rysy systému\\
- oba kopírují základní odezvu systému chemotaxe a zobrazují přesnou adaptaci\\
\indent - fine-tuned - závisí na přesné vyváženosti rozdílných biochemicých parametrech\\
\indent - robustní - pro širší rozptyl parametrů\\

\subsection{Fine-tuned model}
- zjednodušený model teoretického modelu chemotaxe\\
- receptorový komplex X může být metylován $X_m$ v rámci akce CheR, demetylován CHeB\\
- ignorování přesného počtu metylových skupin na receptor, skupina všech metylovaných receptorů do jedné proměnné $X_m$\\
- jen metylovanýreceptor je aktivní, aktivita $a*0$ na metylovaný receptor\\
- takovýto model ztrácí přesnou adaptaci, když $A_0$ (ustálený stav bez atraktantu) neni rovno $A_2$ (ustálený stav s atraktantem)\\
\indent - zde změna v úrovni proteinu CheR o 20$\%$ - trojásobný rozdíl v ustálených aktivitách s a bez\\
\indent \indent ligandu\\
\indent - přesná adaptace je zde fine-tuned vlastnost\\

\subsection{Barkai-Leiblerův robustní mechanismus pro přesnou adaptaci}
- mechanismus pro přesnou adaptaci na širokém rozptylu parametrů\\
- model má několik míst metylace, dalších detailů, reprodukuje mnoho pozorování na dynamiku systému chemotaxe\\
- robustnost přesné adaptace závisí na tom, že CheB pracuje pouze na aktivních receptorech a nedemetyluje receptory, které jsou neaktivní\\
\indent - nezbytné pro robustní adaptaci, není nereálné\\
\indent - povolením malého množství $\epsilon$ pro CheB provádět akce na neaktivních receptorech způsobí\\
\indent \indent ztrátu přesné adaptace faktorem $\epsilon$

\subsection{Robustní adaptace a integrální zpětná vazba}
- speciální zpětná vazba\\
\indent - míra demetylace je úměrná přímo aktivitě víc než jakékoliv jiné entitě\\
\indent - negativní zpětná vazba působí přímo na promměnou, která má být kontrolována\\
\indent - souvisí s inženýrským řídícím principem \textbf{integrální zpětné vazby}\\
\indent \indent - řízení signálem, který v čase integruje chybu mezi výstupem a žádaným výstupem\\
\indent \indent - garantování navedení do chtěného stavu i přes změny v parametrech systému\\
\indent \indent - často se ukazuje jako jedinné možné robustní řešení tohoto problému\\
\\
Experimenty ukázaly, že přesná adaptace je robustní, zatímco ustálený stav aktivity a adaptační časy jsou fine-tuned

\section{Individualita a robustnost v bakteriálních chemotaxích}
- identické buňky mají rozdílný charakter, jak provádějí chemotaxi\\
\indent - některé nervóznější a přemetují častěji, jiné klidnější\\
- individuální charakteristiky buněk trvají desítky minut\\
- čas adaptace na atraktant je také individuální\\
- ustálená frekvence přemetů a je inverzně korelovaná s adaptačním časem buňky\\
- obvod bakteriální chemotaxe má design takový, že klíčová vlastnost, jako je přesná adaptace, je robustní s ohledem na rozdílnost v úrovni proteinů \\ 


\chapter{Robustní vzorování ve vývoji}
- vývoj - proces kdy se z vajíčka stává mnohobuněčný organismus - mnoho rozdělení na vznik embrya\\
- všechny buňky stejný genom\\
\indent - kdyby všechny vyjadřovaly stejný protein, tělo by bylo beztvré a tvořené identickými buňkami\\
\indent - třeba tedy udělit osud, rozdíl v proteinech, které vyjadřují\\
- \textbf{morfogeny} - signálové molekuly (často proteiny)\\
\indent - k vytvoření prostorového vzoru je třeba poziční informace, kterou nesou jejich gradienty\\
\indent - jejich produkce v daném zdrojovém místě, rozptyluje se do regionu který bude paternován\\
\indent \indent - \textbf{pole}\\
\indent - koncentrace vysoká v blízkosti zdroje, snižuje se vzdáleností\\
\indent - buňky v poli jsou ze začátku identické a můžoucítit morfogen pomocí receptorů\\
\indent \indent - navázání morfogenu na receptor, aktivace signálových cest, exprese sady genů\\
\indent \indent \indent - jaké geny vyjadřovány a osud buňky závisí na koncentraci morfogenu\\
- Model Francouzské vlajky - koncentrace morfogenu M(x), zdroj x=0\\
\indent - buňky cítí koncentraci vyšší hraniční hodnota $T_1$ - osud A\\
\indent - nižší než $T_1$ a vyšší než hranici $T_2$ - osud B\\
\indent - nižší než $T_2$ - osud C\\
\indent - 3-regionální vzor - normálně více než 3 osudy\\
- komplexní prostorové vzory se neformují najednou, sekvenční proces\\
- když hrubý vzor vytvořen, buňky vy regionech můžou skrývat další morfogeny pro generování jemnějších subvzorů\\
- některé vzory optřebují průsečík 2+ morfogenů - vznik složitých prostorových rozloženítkání\\
- vývojová transkripční síť - sekvenční regulace genů během vzorovacího procesu\\
- vzorování gradientů morfogenů díky rozptýlení molekul cítěných biochemickými obvody\\
\indent - citlivost vzorů na různost parametrů?\\
\indent - velmi robustní na s ohledem na širokou škálu genetických a prostorových poruch\\
\indent - nejčastěji se měnící parametr je produkční míra proteinů\\
\indent - změnou míry produkce morfogenu často vede k malé změně velikostí a pozic formovaného regionu\\

\section{Exponenciální profily morfogenů nejsou robustní}
- nejjednodušší meganismus - morfogen produkován ve zdroji x=0 a rozptyluje se do okolí s identickými buňkami\\
- degradace $\alpha$ - kombinací s rozptylem to vede k profilu morfogenu s klesající exponenciálou\\
- vzdálenost rozpadu $\lambda$ - vzdálenost, kterou po uražení morfogenu v poli začne degradovat\\
- čím větší konstanta rozptylu D, tím mneí je degradační míra $\alpha$\\
- rozpad je dramatický - vzdálenost $10\lambda$ - koncentrace 5$\%$ a $5.10^{-5}$ původní hodnoty\\
- $\lambda$ - typická velikost regionu, která může být vzorována daným gradientem\\
- problém, když je produkční míra zdroje morfogenu porouchaná ($M_0$)- dosažení jiné hranice, jiného osudu\\
- tento proces nevysvětluje robustnost pozorovanou ve vývojovém vzorování\\

\section{Zvýšená robustnost se samozlepšnou degradací morfogenu}
- použití nelineární míry degradace\\
- posun $\delta$ (rozdíl mezi originální a posunutou hranicí) v morfogenním profilu po změně $M_0$ je v prostoru jednotný\\
\indent - všechny regiony jsou posunuty o stejnou vzdálenost jako $M_0$\\
- cílem zvýšit robustnost - co nejmenší posun $\delta$ při změně v $M_0$ na $M_0'$\\
\indent - míra rozpadu blízko x=0 co největší, dosažení $M_0'$ s malým posunem\\
\indent \indent - lze klesajícím vzdáleností $\lambda$ - neakceptovatelné\\
\indent \indent \indent - rozsah morfogenu, tedy velikost vzoru, je silně redukována\\
\indent \indent \indent - potřeba najít profil s velkým dosahem a robustností\\
\indent \indent \indent \indent - rychlý rozpad kolem x=0 k dosažení robustnosti při změně v $M_0$\\
\indent \indent \indent \indent - pomalý rozpad na velkém x k dosažení velkého rozsahu M\\
- nelineární samozlepšená degradace\\
\indent - vzpětnovazební mechanismus dělající degradační míru M rostoucí s koncentrací M\\
\indent - samozlepšená degradace povoluje ustálený stav profilu morfogenu s mírou nestejnoměrného rozpadu\\
\indent - profil se rozpadá rychle kolem zdroje - robustnost ve změnách produkce morfogenu\\
\indent - pomalý rozpad dále od zdroje - vzorování ve velkém rozsahu\\

\section{Motivy sítě, které provádí degradační zpětnou vazbu pro robustní vzorování}
- navázání morfogenu na receptor, změna exprese genu\\
\indent - 2 typy smyček v rozdílných vývojových procesech\\
- zpětnovazební smyčka, kde receptor R zlepšuje degradaci M\\
\indent - morfogen vážící se k R spouští signalizaci, zvýšení exprese R\\
\indent - degradace M způsobena zvedáním vazby morfogenu do receptoru a jeho poruchou v rámci buňky\\
\indent - M zlepšuje produkci R, R zlepšuje míru endocytózy a degradace M\\
- symčka, když R potlačuje degradaci M\\
\indent - navázáním M na R se spouští signalizace, represe exprese R\\
\indent - R potlačuje navázáním se degradaci M a potlačuje protein degradující M (mimobuněčná\\
\indent \indent protáza) nebo represuje expresi protázy\\
- v obou případech M zvyšuje vlastní míru degradace, která podporuje robustní vrozování ve velkém rozsahu\\


\chapter{Kinetické korektury}
- kinetická korektura - principy s vysokou přesností, které jsou využívány v rozmanitých molekulárních rozpoznávacích systémech\\
- čtení genetického kódu během translace - řerězec je syntetizován přidáváním monomeru v každém kroce\\
\indent - typ monomeru je zvolen informací kódovanou v šabloně - při translaci je to v mRNA\\
\indent - kvůli teplotnímu šumu je občas přidán nesprávný monomer\\
\indent \indent - kinetická odezva je hlavní nástroj na redukci míry chybovosti\\ 
\indent \indent \indent - výrazně přesnější než obyčejné rozvnovážné porovnávání mezi monomery\\
- zavedení do poznávacího problému v imunitním systému\\
\indent - jak imunitní systém rozpozná proteiny jdoucí ze škodlivých mikrobů přes přítomnost velmi\\
\indent \indent podobných proteinů jdoucích z ze zdravých buněk těla\\
\indent - použití malý rozdíl ve slučivosti proteinového ligandu pro vytvoření velmi přesného rozhodnutí\\
- strategie Picassovy místnosti - půlka ho má ráda (v místnosti 10min), půlka ne(v místnosti 1min)\\
\indent - otevření mísnosti, vpuštění lidí, zavření, otevření východu z místnosti\\
\indent - po minutě odchází co nemají rádi a zůstávají pouze ti co jo - výrazně více než 10x víc\\
\indent \indent - v případě stálého průchodu by to bylo 10x více\\
\indent - použití podobné strategie, jwjí reakce jsou nereverzivní, nevyvážená\\

\section{Kinetická korektura genetického kódu může redukovat míru chybovosti molekulárního rozpoznávání}
- při translaci ribozomy produkují proteiny spojováním se s aminokyselinami jedna po druhé do řetězce\\
\indent - pořadí informací kódovanou mRNA\\
\indent - aminokyseliny kódované kodonem - série 3 písmen na mRNA\\
\indent - mapování mezi 64 kodony a 20 aminokyselinami je genetický kód\\
- při tvorbě proteinu musí kodony být čteny a odpovídající aminokyseliny přivedeny do ribozomu\\
\indent - aminokyselina je přivedena připojena k dané tRNA molekule\\
\indent - tRNA má 3 písmenné rozpoznávací komplementární místo\\
\indent \indent - páruje se se sekvencí kodonů pro aminkoyselinu na mRNA\\
\indent \indent - pro každý kodon je tRNA, která specifikuje aminokyselinu v genetickém kódu\\
- kodon musí rozpoznat a připojit se na správnou tRNA - termální šum - chybovovst\\
\indent - ze 100 aminokyselin bývá 1\% šance, že je 1 špatně - více by bylo fatální\\
\indent \indent - nepřijatelná frakce buňky\\

\subsection{Rozvnováha vázání nevysvětluje přesnost translace}
- nejjednodušší model pro tento porovnávací proces - produkce míry chybovosti 100x větší než ta pozorovaná\\
- kodon C na mRNA v ribisomu, který kóduje aminokyselinu a je přidáván na konec proteinového řetězce\\
\indent - míra správného navázání na C - $c$, C se váže na $c$ s on-mírou $k_c$\\
\indent - tRNA se odváže od kodonu s off-mírou $k_c'$\\
\indent - tRNA navázané - pravděpodobnost, aminokyselina navázaná na tRNA bude kovalentně\\
\indent \indent spojena s rostoucím translatovaným proteinovým řetězcem\\
\indent - osvobozené tRNA se odváže od kodonu a ribozom se přesune k dalšímu kodonu v mRNA\\
\indent - počet správných tRNA je přibližně stejný jako počet nesprávných tRNA\\
\indent - \textbf{míra chybovosti} - poměr správného a špatného začlenění aminokyseliny\\
\indent - špatné tRNA se odváže od mnohem rychleji než správná tRNA - slaší vazba - Picassova místnost\\
- rovnováha vázání provede rozlišování jen tak dobré, jako je poměr mezi správnými a chybnými cíly\\

\subsection{Kinetická korekce může dramaticky snížit míru chybovosti}
- tRNA, po navázání kodonu, projde chemickou modifikací - $c$ naváže C a změní se v $c^*$\\
\indent - modifikovaná tRNA - $c^*$ - může odpadnou od kodonu nebo darovat aminokyselinu prodlužujícímu\\
\indent \indent se řetězci\\
\indent - prakticky nereverzivní - zdá se zbytečné, může dojít ke ztrátě správné tRNA\\
\indent - design generující vysokou přesnost\\
\indent - $c^*$ nabídne 2. rozlišovací krok\\
\indent \indent - jednou modifikovaná špatná tRNA může odpadnout od kotonu, ale ne už se zpátky\\
\indent \indent \indent navázat\\
\indent \indent \indent - nereverzivní reakce funguje jako Picassova místnost s 1 dveřmi\\
\indent - povolení 2 nezávislých rovnovážných procesech rozpoznávání\\
\indent \indent - druhý pracuje na výstupu prvního - míra chybovosti rovna 2. odmocnině\\
- míra chybovosti 1/10000 - podobné pozorované chybovosti\\
- dosáhnutí větší míry přesnosti spojením dohromady více nereverzivních korekčních procesů\\

\section{Rozeznávání vlastního a nevlastního imunitním systémem}
- trochu jiné vysvětlení kinetické korekce - časová zpoždění - kinetická korekce v imunitním systému\\
- imunitní systém monitoruje tělo - nebezpečné patogeny\\
\indent - při zpozorování vypočítá a zmobilizuje odpovídající odpověď\\
\indent - imunitní sytém je obrovská kolekce buněk komunikujících mezi sebou nesčetně způsoby\\
\indent - protilátky - protein s designem, aby se vázal na cizí proteiny vytvářené patogeny - antigen\\
\indent - T-cells - buňky skenující tělo, hledání antigenů\\
\indent \indent - receptory z daných protilátek proti cizímu proteinovému antigenu\\
\indent \indent - k poskytnutí informace T-buňce, každá buňka těla má zlomky proteinů na povrchu\\
\indent \indent \indent - proteiny ve vyhrazených proteinových komplexech na povrchu buňky - MHCs\\
\indent \indent - cílem eliminovat infikovanou buňku\\
\indent \indent - rozpoznání antigenu receptorem, cizí fragment proteinu v MHC na proteinu spustí signál\\
\indent \indent \indent transdukční kaskádou v buňce, T-buňka zabije buňku prezentovanou cizím peptidem\\
\indent \indent - T-buňky ničí buňky těla - autoimunitní choroba\\
\indent \indent - míra chybovosti je méně než $10^{-6}$, ačkoliv spříznění antigenů je často jen 10x větší než\\
\indent \indent \indent spříznění vlastních buněk\\

\subsection{Rovnováha navazování není vysvětlení nízké míry chybovosti imunitního rozpoznávání}
- receptory na T-buče tak, aby rozpoznaly specifický cizí protein - přesný ligand $c$\\
- $c$ se váže na receptor s velkým spřízněním\\
\indent - receptory dále vystaveny mnoha vlastním proteinům - menší spříznění\\
- míra chybovosti vedoucí k poměru spříznění přesných a nepřesných cílů vynásobená poměrem jejich koncentrací\\
- spříznění přesných a nepřesných cílů je velmi podobné, míra nerozpoznání je příliš vysoká\\
- přesný ligand stráví více času navazováním spojení s receptorem\\
- nepřesné ligandy mají cca 10x menší spříznivost než přesné, nepřesných (zdravé) je ale často více než přesných\\
- míra nepřesnosti je zde větší než 0.1 $\rightarrow$ výrazně více než pozorovaná $10^{-6}$ a méně\\

\subsection{kinetická korekce zvyšující přesnost T-buněčného rozpoznávání}
- zesílení malého rozdílu na velký rozdíl pro míru rozpoznávání\\
- po navázání ligandu dochází k sérii kovalentních modifikací receptoru - fosforylace na početných místech\\
\indent - konzumují energii a jsou zadrženy od termální rovnováhy\\
- při modifikaci naváže receptor v buňce pár proteinových partnerů\\
- aktivace signálové cesty až po všech modifikacích\\
- kinetická korekce závisí na zpoždění, které je těmito kroky vytvořeno\\
\indent - pouze ligandy, které se udrží déle, mají šanci aktivovat T-buňku\\
\indent - čím větší je zpoždění, tím je větší počet událostí navazování přesného ligandu, který se odpojí\\
\indent \indent před začátkem signalizace\\
\indent \indent - zvýšení zpoždění může mít za následek ztrátu citlivosti\\
\indent \indent \indent - tolerováno díky značně zlepšenému rozlišování mezi přesnými a nepřesnými ligandy\\
\indent - modifikace musí zmizet dříve, než se odpojí ligand, aby se mohl hned připojit jiný\\
\indent - čím je ligand připojen déle, tím je větší šance, že bude spouštět signalizaci\\
- proces v T-buňkách není unikátní, děje se skoro všude v receptorech v savčích buňkách\\
- poskytuje robustnost proti chybnému ropoznání pozadí různých molekulo v oraganismu\\

\section{Kinetická korekce se může objevit v různých rozpoznávacích procesech v buňce}
- hlavním znakem kinetické korekce je existence nerovnovážná reakce při rozpoznávacím procesu, která formuje mezistav, který vytváří zpoždění po navázání ligandu\\
- systém musí pracovat mimo rozvnováhu, aby ligandy nemohly obejít zpoždění znovunavázáním přímo na již modifikovaný stav\\
- další výskyty - vazba DNA opravným proteinem, rekombinace proteinů\\
- rozpoznávací protein A se naváže na poničená vlákna DNA - vyšší spřízněnost k poničené DNA než normální DNA\\
\indent - potom modifikace (fosforylace), naváže další proteiny B a C, které nahradí poničená vlákna\\
\indent \indent DNA\\
\indent - modifikační krok proteinu A může zabránit chybému orzpoznání DNA jako zničené\\
- navazování aminokyseliny na dané místo tRNA\\
\indent - speciální enzym rozpozná tRNA a její specifickou aminokyselinu a kovalentně se připojí\\
\indent - připojení špatné aminokyseliny na tRNA by mohlo vést k začlenění chybné aminokyseliny v\\
\indent \indent translatovaném proteinu\\ 
\indent - chybovost kolem $10^{-4}$\\
\indent \indent - docíleno vysokoenergetickým mezistavem, ve kterém enzym, který spojuje aminokyselinu\\
\indent \indent \indent k tRNA, nejdříve naváže oba reaktanty, modifikuje tRNA, a až pak formuje\\
\indent \indent \indent kovalentní vazbu mezi nimi\\


\chapter{Optimální design genového obvodu}
- zde jednoduchá aplikace teorie přirozeného výběru v genovém obvodu\\
- optimalita buněčného obvodu?\\
\indent - mnoho mutací zhorší výkon\\
\indent - funkce zdatnosti - maximalizace - potíž, že ji v reálném světě neznáme\\
\indent \indent - není tedy třeba optimalizovat, ale jen dojít "uspokojivého" výsledku\\
- optimalita - idealizovaný předpoklad, dobrý startovací bod pro generování testovacích hypotéz genového obvodu\\
\indent - zde rozebrání nejjednodušších systémů, dá se vytvořit fenomenologický popis základních sil, \\
\indent \indent které působí při hře přirozeného výběru\\
- jednoduchá situace, bakterie roste konstantním prostředí, které je konstantně doplňováno\\
\indent - lze definovat funkci zdatnosti - základ na míře růstu organismu\\
\indent - bekterie s nejrychlejší mírou růstu může převzít populaci\\
\indent \indent - zajištěno, že růstová výhoda je dostatečně velká na to, aby překonala náhodný efekt\\
\indent \indent \indent genetického driftu\\
\indent \indent - evol. výběr podmíněný růstem v konstantním prostředí vede k maximalizaci míry růstu\\
- detailní příklad - funkce zdatnosti laktózy (lac) v E.coli\\

\section{Optimální úroveň exprese proteinu při konstatních podmínkách}
- \textbf{funkce zdatnosti} - f - velikost k optimalizaci\\
\indent - v příznivém prostředí $f$ míra růstu buňky\\
- pár bakterií, růst počtu N exponenciálně, dokud nejsou příliš hustě
\begin{equation}
N(t)=N(0)e^{ft}
\end{equation}
- souboj 2 druhů s různým $f$ - ten s vyšší vyhraje a získá celý prostor\\
- co určuje míru exprese proteinu? - $lac$ system E.coli\\
\indent - kódování proteinů LacZ - zozklad cukr laktózy - zdroj energie a uhlíku\\
\indent \indent - cca 60000 kopií LacZ na buňku\\
\indent - úroveň exprese je vybrána maximalizací funkce zdatnosti\\
\indent - nejjednodušší prostředí s konst. podmínkami - prostředí s konstatní koncentrací cukr laktózy\\
- \textbf{náklady} produkce proteinu LacZ, \textbf{prospěch}, který to přivádí buňce\\

\subsection{Prospěch z proteinu LacZ}
- prospěch - relativní nárůst míry růstu díky akci proteinu\\
- prospěch úměrný míře, kdy LacZ rozkládá substrát - laktózu\\
- míra enzymu LacZ je dobře popsána - LacZ rozkládá laktózu v míře úměrné počtu kopií proteinu (Z), vynásobené saturační funkcí koncentrace laktózy (L)\\
\begin{equation}
b(Z,L)=\frac{\delta ZL}{K+L}
\end{equation}
\indent K ... Michaelisova konstanta\\
\indent $\delta$ ... maximální výhoda míry růstu na protein LacZ při saturující laktóze\\
- prospěch roste lineárně s úrovní proteinu Z\\
- experimentálně vyvolávačem IPTG - chemický analog laktózy - exprese proteinu Lac - není metabolizován buňkami\\
\indent - nerokuje žádný prospěch pro sebe\\

\subsection{Náklady proteinu LacZ}
- experimentálně změřeno vyvoláním exprese proteinu LacZ na různé úrovně pomocí vyvolávače IPTG při absenci laktózy\\
\indent - IPTG vynakládá pouze náklady produkce proteinu, nedává žádný prospěch, protože nemůže\\
\indent \indent být využit buňkami\\
- exprese LacZ snižuje míru růstu buňky\\
\indent - náklady rovné míře růstu jsou nelineární funkce Z\\
\indent \indent - čím více produkováno, tím je vyšší náklad na každou další buňku\\
\indent \indent - produkce proteinu závisí na zrdojích buňky, které se tvorbou redukují i pro další užitečné\\
\indent \indent \indent proteiny\\
\indent - míra růstu buňky závisí na vnitřních zrojích R\\
\begin{equation}
f\sim\frac{R}{K_R+R}
\end{equation}
- produkce Z je pro buňku břemeno - produkce mRNA, sytetizace aminkyselin a spojení k formě Z\\
\indent - redukce interních zdrojů R\\
\indent - redukce míry růstu začne divergovat, když je vyprodukováno příliš Z a R se začne vyčerpávat\\
\begin{equation}
c(Z)=\frac{\eta Z}{1-Z/M}
\end{equation}
- vytvořeno pár kopií proteinu - náklady skoro lineární\\
- náklady rostou prudce když jeZ srovnatelné s horním limitem exprese (M), když začne značně zasahovat do ostatních nezybtných proteinů\\
\indent - reálně se proteiny nedostávají příliš blízko hranici Z=M, kde by nákladová funkce divergovala\\
- relativní redukce míry růstu komplentně vyvolaným systémem $lac$ je kolem 4,5\%\\

\subsection{Funkce zdatnosti a optimální míra exprese}
- funkce zdatnosti - rozdíl v nákladech a prospěchu\\
\begin{equation}
f_L(Z)=b(Z,L)-c(Z)
\end{equation}
- maximalizace:
\begin{equation}
Z_{opt}=M(1-\sqrt{\frac{\eta(K+L)}{\delta L}})
\end{equation}
\indent - čím více laktózy v prostředí, tím vyšší odhadovaná optimální úroveň proteinu\\
\indent \indent - žádná laktóza - $Z_{opt}$ = 0 - protein by si rokoval jen náklady a nebyl by žádný prospěch\\
\indent \indent - normálně kolem $Z_{opt}$ = 60000/buňka\\
\indent \indent - když náklady převyšují prospěch, neni důvod protein vyrábět - málo laktózy v prostředí\\
\indent \indent \indent - více generací v takovém prostředí - ztráta genu kódujícího LacZ\\

\subsection{Experiment laboratorní evoluce ukazuje, že buňky dosáhnou optimální úrovně LacZ za pár set generací}
- růst buněk E.coli ve zkumavkách s danou úrovní laktózy\\
\indent - úroveň dostatečná, aby garantovala plnou indukci erxprese LacZ\\
- každý den byla 1/100 buněk z každé zkumavky přendána do nové s čerstvým prostředím - \textbf{sériové ředění}
- růst buněk do stacionární fáze, 1/100 jinam a pořád dokola\\
- předpokládané úrovně LacZ dosaženy po několika stech generacích\\

\section{Regulovat nebo ne: optimální regulace v proměnných prostředích}
- některé geny se regulují, některé ne - kdy se to vyplatí?
- nekonstatní prostředí - produkt genu Z dává prospěch buňce, když je splněna podmínka prostředí $C_Z$\\
- regulace - Z produkováno při podmínce $C_Z$ a když je potřeba\\ 
\indent - náklady - produkce a udržování regulačního systému\\
\indent - \textbf{poptávka} po Z - pravděpodobnost $p$, že organismus ukáže podmínku $C_Z$
\indent \indent - když $C_Z$ nadbytečné - pravděpodobnost 1-p\\
- funkce zdatnosti - prospěch, nároky\\
\indent - s produkcí Z klesá míra růstu kvůli břemenu syntézy a údržby - nároky $c$\\
\indent - prospěch buňky z akce Z - výhoda míry růstu - $b$\\
- 1. organismus - protein Z neregulován, produkován konstantně za všech podmínek\\
\indent - konstitutivní exprese\\
\indent - neustálá produkce Z - prospěch pouze ve zlomku doby, když je Z požadováno\\
\begin{equation}
f_1=pb-c
\end{equation}
- 2. organismus - regulační systém, Z produkováno při podmínce $C_Z$ - požadovaná podmínka\\
\indent - šetří produkcí, která je jen, když je třeba - platí nároky $c$ je zlomek času\\
\indent - nese nároky $r$ na trovbu a údržbu regulačního systému R\\
\begin{equation}
f_2=pb-pc-r
\end{equation}
- 3. organismus - nemá regulátor ani gen na produkci Z\\
\begin{equation}
f_3=0
\end{equation}
- výběr regulace, když má organismus 2 největší zdatnost, neregulovaný design, když ma největší zdatnost 1. organismus\\

\section{Výběr prostředí motivu sítě dopředné smyčky}
- koherentní FFL - zpoždění na vstupu při ON, ne při OFF\\ 
- rozšířený v transkripčních sítích, ne každý gen je zahrnut v FFL\\
- transkripční síť E.coli - 40\% známých genů regulováno 2 vstupy je regulováno pomocí FFL\\
\indent - 60\% regulováno jednoduchým 2-vstupovým designem\\
- zjednodušená analýza prospěchu-nákladů pro výběr genového obvodu v daném kolísajícícm prostředí\\
\indent - prostředí - časově závoslý profil vstupního signálu v přirozeném místě výskytu organismu\\
- předpoklad - přítomnost systému se vstupním signálem $S_x$ po dobu D\\
- funkce zdatnosti založena na proprospěchu a nákladech proteinu Z\\
\begin{equation}
\phi(D)=\int_0^Df(t)dt
\end{equation}
- krátké pulsy mají určitý efekt na růst - vedou k redukci zdatnosti\\
\indent - redukce zdatnosti - vstupní impuls kratší než kritická doba pulsu ($D_C$), protein Z nemá čas\\
\indent \indent nahromadit se tak aby prospěch předčil náklady produkce\\
- když je zdatnost zredukována expresí proteinu Z v reakci na krátý puls, začne být obvod výhodný\\
\indent - C1-FFL přesně tohle umí - exprese Z až po uplynutí zpoždění\\
- filtrace signálů kratších než $T_{ON}$ - zabraňuje redukci růstu pro krátké pulsy\\
- nevýhody - produkce Z po zpoždění - vynechává potenciální prospěch pulsu - někdy více škody než užitku\\
- analýza rozložení signálů v daném prostředí\\
\indent - pravděpodobnostní rozložení trvání vstupního pulsu - P(D)\\
\indent - oddělení pulsů - kyždý při příchodu začání na počáteční hodnotě Z=0 - i Y=0 v případě FFL\\
\indent - průměrná zdatnost - integrace zdatnosti na puls přes rozložení pulsu\\
\begin{equation}
\Phi=\int P(D)\phi(D)dD
\end{equation}
\indent - design s vyšší průměrnou zdatností má výhodu výběru\\
- FFL není vybráno v prostředích s exponenciálním rozdělením pulsů\\
- naopak výběr v prostředích s bimodálním rozložením pulsů\\
- zpoždění u FFL ideálně takové, aby bylo přesně rovno krátkým pulsům\\
- odfiltrování pulsů bez prospěchu a zároveň minimální negativní dopad na zdatnost při dlouhých pulsech\\



\end{document}